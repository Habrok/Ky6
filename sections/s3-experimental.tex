Усі зразки, які досліджувались у даній роботі, були монокристалами карбіду кремнія кубічної модифікації (3C-SiC) n-типу провідності. Зразки вирощувались з використанням методу паро-фазового осадження, шляхом розкладу метилтрихлорсилану при температурі Т~=~1700~K у атмосфері водню та його подальшого осадження на графітових стрижнях. Зразки не легувались, концентрація донорів азоту, не спеціально утворених під час росту кристала складала 10$^{17}$~cм$^{-3}$. Типові розміри усіх досліджуваних зразків складали 4~х~3~х~0.7~мм. Опромінення нейтронами відбувалося у ядерному реакторі при кімнатній температурі, доза опромінення складала 1~х~10$^{19}$~см$^{-2}$. Зважаючи на високу проникність нейтронів у SiC, розподіл дефектів можна вважати гомогенним \citep{neut}.\\
\indentСпектри ЕПР вимірювались за допомогою спектрометрів Х-діапазону при температурах 300~К, 77~К та 4.2~К для випадку обертання магнітного поля у площинах ($1\bar{1}0$) та (111). Температурний відпал відбувався у атмосфері інертного газу гелію, у діапазоні температур Т$_{ann}$~=~200~..~1100~K із кроком 100~К та часом витримки 5~хвилин. Термопара дозволяла контролювати значення температури у межах 1$\degree$.
