\begin{minipage}{\textwidth}\mbox{}\\У кристалах 3C-SiC, опромінених нейтронами, відпал при T~=~900~\degree C привзодить до повного зникнення вакансій кремнію, одночасно з чим спостерігається поява нового парамагнітного центра, названого Ку6. Даний центр має значення ефективного спіна S = 1/2 та описується аксіальним g-фактором з напрямком головної осі тензора уздовж вектора <111> та головними значеннями g$_\parallel$~=~2.0024 та g$_\perp$~=~2.0029. Центр Ку6 спостерігаються при температурах Т~$<$~140~K та має чіткі лінії анізотропної надтонкої взаємодії з розщепленням А$_\parallel$~=~23.9$\cdot$10$^{-4}$~см$^{-1}$ та А$_\perp$~=~19.3$\cdot$10$^{-4}$~см$^{-1}$. Спираючись на властивості надтонкої ваємодії, процеси у системі дефектів кристала, пов'язані з відпалом та теоретично розраховану енергію утворення $[???]$, центр Ку6 було пов'язано V\textsubscript{C}C\textsubscript{Si}\textsuperscript{-}.
\end{minipage}

