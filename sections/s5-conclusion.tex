У даній роботі проведено дослідження кристалів 3C-SiC n-типу, опромінених реакторними нейтронами, спостережено та описано новий парамагнітний центр зі спіном S~=~1/2, названий Ку6. Аналіз його спектрів ЕПР, кінетики відпалу та поівняння отриманих результатів з теоретичними розрахунками потенційних бар'єрів та положень енергетичних рівнів, дають підґрунтя для побудови моделі даного дефекту та інтерпретувати його, як комплекс вакансія вуглецю - антисайт вуглецю у нейтральному зарядовому стані [V\textsubscript{C}C\textsubscript{Si}]\textsuperscript{0}.
