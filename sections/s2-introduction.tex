З-поміж різноманіття напівпровідникових матеріалів, карбід кремнію (SiC) посідає особливе місце, що визначається чудовими властивостями даного матеріалу. SiC характеризується значними температурною, радіаційною та хімічною стійкістю, що робить його цікавим матеріалом для виготовлення приладів, які можуть використовуватись в умовах відкритого космосу, у ядерній та хімічній промисловостях, тощо.\\
\indentСеред понад двухсот політипів, кубічний 3C-SiC характеризується найкращими електронними та механічними властивостями \citep{choy1}. Проте, зважаючи на більш прості умови вирощування гексагональних кристалів, саме вони набули широкого промислового використання і, як наслідок, добре досліджені (наприклад, \citep{hex1} \citep{hex2}). З іншого боку, сучасні досягнення у ґалузі технологічних процесів вирощування кубічних зразків, насамперед тонких плівок \citep{epilay}, повертають інтерес до промислового використання приладів, створених на їхній основі.\\
\indentОпромінення кристала частинками різного роду та енергії призводить до пошкодження ідеальності його ґратки та утворення різних точкових дефектів. Точкові дефекти, у свою чергу, безпосередньо впливають на електронні та фізичні властивості кристалу. Таким чином, зважаючи на сфери застосування 3C-SiC, інформація про власитивості дефектів, утворених у ньому під дією опромінення, та їхні перетворення, містять значний практичний та науковий інтерес. Особливе місце посідають процеси, котрі відбуваються у кристалі під час відпалу.\\
\indentВідомо, що основними дефектами у щойноопромінених кристалах 3C-SiC є від'ємнозаряджена вакансія кремнію \citep{t1} та нейтральна дивакансія \citep{ky5}. Вакансії кремнію спостерігались у кристалах, опромінених електронами з енергіями 1-2~МеВ, протонами, нейтронами, тощо, мають спін S~=~3/2 та описується ізотропним g~=~2.0029 \citep{t1}. У той же час, дивакансія спостерігається лише у зразках, опромінених нейтронами \citep{ky5}, має спін S~=~1 і характеризується аксіальним тензорам тонкої взаємодії D~=~443$\cdot10^{-4}$ см$^{-1}$ та ізотропним g-фактором g~=~2.003. У результаті відпалу кристалу, при температурі Т\textsubscript{відп}~=~900~K вищезазначені дефекти зазнають перетворень і майже повністю відпалються. У той же час, спостерігається поява спектра ЕПР, який належить новому дефекту зі спіном S~=~1/2, названому Ку6. Дана робота присвячена дослідженню властивостей цього дефекту, та механізмів його утворення.
