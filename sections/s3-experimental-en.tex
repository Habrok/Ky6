All samples were grown by vapor-phase process.  Thermal decomposition of methyl trichlorsilane in hydrogen environment at T~=~1700~K and consequent precipitation on graphite needles used to obtain required compound during this process. All samples weren't intentionaly doped and n-type of conductivity had been set by Nitrogen donors, that formed during crystal growth. Concentration of this donors were 10$^{17}$~cm$^{-3}$. Typical sample dimentions were 4~x~3~x~0.7 mm. Neutron irradiation were carried out in nuclear reactor at room temperature with dose of 1~x~10$^{19}$~cm$^{-2}$ exposed to each sample. Due to high neutron penetration ability of SiC \citep{neut}, the distribution of generated defects can be considered as homogeneous.\\
\indent X-band spectrometers Radiopan SE/X and Bruker E550 were used to measure EPR spectra. Measurements were carried out at the 300~K, 77~K and 4.2~K with magnetic field rotated in ($1\bar{1}0$) and (111) planes. For studying of transformations of defects, samples were annealed in inert environment of Helium gas in the temperature range from Т$_{ann}$~=~200~to~1100~K with step 100~K and exposure 5~mins. Temperature control performed with thermopair that gave precision about $\pm$1$\degree$.
