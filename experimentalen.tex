All samles were grown by vapor-phase process.  Thermal decomposition of methyl trichlorsilane in hydrogen environment at T~=~1700~K and consequent precipitation on graphite needles used for produce needed compound during this process. All samples weren't doped and n-type of conductivity was defined by Nitrogen donors, formed during crystal growth. Concentration of this donors were 10$^{17}$~cm$^{-3}$. Typical sample dimentions were 4~x~3~x~0.7 mm. Neutron irradiation were carried out in nuclear reactor at room temperature and resulting dose exposed to each sample were 1~x~10$^{19}$~cm$^{-2}$. Due to high neutron penetration ability of SiC \citep{neut}, the distribution of generated defects can be suggested as homogeneous.\\
\indent EPR spectrta were mesured with X-band spectrometers Radiopan SE/X and Bruker E550 at 300~K, 77~K and 4.2~K. Magnetic field rotated in ($1\bar{1}0$) and (111) planes.Thermal annealing carried out in inert environment of Helium gas in temperature range Т$_{ann}$~=~200~..~1100~K with step 100~K and exposure 5~mins. Temperature control performed with thermopair that gave precision about $\pm$1$\dergee$.
