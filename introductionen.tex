Due to its remarkable electronic and physical properties, silicon carbide takes a substantial place among all semiconducting materials. High thermal, radiational and chemical endurance makes it a significant substance for developing electronic devices that can be used in severe environments like space or nuclear reactors.\\
\indent There are over 200 polytypes of SiC, but cubic 3C-SiC has simplest crystal structure and highest electron mobility among all \citep{choy1}. Unfortunately, due to complexity in growth conditions of 3C-SiC it has not reached industrial implementation, simultaneously the greater commercial success have been reached for hexagonal crystals and therefore they much better studied during past decades (see \citep{hex1, hex2} and references therein). Recent progress in 3C-SiC growth, especially thin films \citep{epilay}, raised up interest for industrial use cubic crystals as basis for industrial electronics.\\
\indent Irradiation of crystal produces high concentrations of various point defects and therefore influences on its electronic properties. Thus perspectives of industrial use of 3C-SiC raises the interest to studying properties of its point defects. Dominant defects in as-irradiated 3C-SiC crystals are negative silicon vacancy \citep{t1} and neutral divacancy \citep{ky5}. Silicon vacancy was observed in crystals irradiated by electrons, protons, neutrons and ions, has spin S~=~3/2 and can be described by isotrop g~=~2.0029. From another hand, divacancy observed only in neutron irradiated samples, has S~=~1 and isotrop fine structure tensor and g-factor: D~=~443$\cdot10^{-4}$ cm$^{-1}$ and g~=~2.003 respectively. Annealing at Т\textsubscript{ann}~=~900~K cause transformations in system of defects that results in almost full disappearence of mentioned above defects. At the same time, new EPR spectrum was emerging. This spectrum was linked to S~=~1/2 paramagnetic defect, named Ky6, and the present paper has been dedicated for elucidate the properties of this defect.
